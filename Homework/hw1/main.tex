% Copyright (c) 2026 Frankie Feng-Cheng WANG. All rights reserved.
% Repository: https://github.com/FrankieeW/
% !TEX program = xelatex



%----------------------------------------------------------------------------------------
% PACKAGES AND OTHER DOCUMENT CONFIGURATIONS
%----------------------------------------------------------------------------------------

\documentclass[
  12pt, % Default font size, values between 10pt-12pt are allowed
]{assignment}

% Set table of contents depth to section level only (no subsection, subsubsection)
\setcounter{tocdepth}{1}
\setcounter{secnumdepth}{2} % Still number sections up to subsubsection, just don't show in TOC

\counterwithin{equation}{section}
\counterwithin{figure}{section}

\usepackage{amsthm} % For proof environment
\newtheorem{lemma}{Lemma}
\newtheorem*{claim}{Claim}
\newtheorem*{corollary}{Corollary}
\newtheorem*{remark}{Remark}
\newtheorem*{definition}{Definition}
\newtheorem*{example}{Example}
\newenvironment{solution}
{\par\noindent\rule{\linewidth}{0.4pt}\par\smallskip\begin{proof}[Solution]}
{\end{proof}\par\noindent\rule{\linewidth}{0.4pt}\par\smallskip}

\usepackage{float}
\usepackage{hyperref}
\usepackage{mathtools}
\usepackage{amssymb}
\usepackage{xcolor}
\usepackage{mathrsfs}
\usepackage{enumitem}
\usepackage{tikz}
\usetikzlibrary{arrows,shapes.geometric,calc}

\newcommand\nb{\addtocounter{equation}{1}\tag{\theequation}}
\newcommand{\subqdash}{%
  \par\vspace{0.8\baselineskip}%
  \noindent\makebox[\linewidth][l]{\leaders\hbox{-}\hfill\kern0pt}%
  \par\vspace{0.8\baselineskip}%
}
\pgfplotsset{compat=1.18}
\renewcommand{\thesection}{Exercise \arabic{section}}

%-------------------------------------------------------------------------------
% ASSIGNMENT INFORMATION
%-------------------------------------------------------------------------------

% Fix for fancyhdr warning: increase headheight and adjust topmargin
\setlength{\headheight}{29.35158pt}
\addtolength{\topmargin}{-17.35158pt}

\title{Assessed Coursework 1} % Assignment title
\author{Frankie Feng-Cheng WANG} % Student name
\email{maths@frankie.wang\\fw225@ic.ac.uk} % Student email
\date{16th February 2026, 1 PM} % Due date
\institute{Department of Mathematics\\Imperial College London} % Institute or school name
\course{MATH70029-Algebraic Geometry} % Course or class name
\lecturer{Matt Booth} % Lecturer or teacher in charge of the assignment

%----------------------------------------------------------------------------------------
\begin{document}
\setcounter{tocdepth}{3}
\maketitle % Output the assignment title, created automatically using the information in the custom commands above
% \tableofcontents

%----------------------------------------------------------------------------------------
% ASSIGNMENT CONTENT
%----------------------------------------------------------------------------------------
\section{}
50 points: 10 + 10 + 10 + 10 + 10.\\
As always, we let $k$ be an algebraically closed field. Some answers may depend on the characteristic of $k$.
\begin{enumerate}
	\item Consider the following polynomials in $k[x,y,z]$:
	      \[
		      f = y^2 - x^2, \qquad g = x^4 - yz, \qquad h = z^2 - x^3 y.
	      \]
	      Find and describe the irreducible components of the varieties $V(f,g)$, $V(f,h)$, $V(f,g,h)$.
	      \begin{solution}
		      Throughout, note that $f = (y-x)(y+x)$.

			      \textbf{$V(f,g)$:}\quad
			      On $V(y-x)$: substituting $y = x$ into $g$ gives $x^4 - xz = x(x^3 - z) = 0$, so $x = 0$ or $z = x^3$.
		      \begin{itemize}
			      \item $x = 0, y = 0$: the $z$-axis $V(x,y) = \{(0,0,t) : t \in k\}$.
			      \item $z = x^3, y = x$: the twisted cubic $C_1 = V(y - x,\, z - x^3) = \{(t, t, t^3) : t \in k\}$.
		      \end{itemize}
		      On $V(y+x)$: substituting $y = -x$ gives $x^4 + xz = x(x^3 + z) = 0$, so $x = 0$ (the $z$-axis again) or $z = -x^3$.
		      \begin{itemize}
			      \item $z = -x^3, y = -x$: the curve $C_2 = V(y + x,\, z + x^3) = \{(t, -t, -t^3) : t \in k\}$.
		      \end{itemize}
			      Each component is irreducible:
			      \[
				      k[x,y,z]/(x,y) \cong k[z],\quad
				      k[x,y,z]/(y-x, z-x^3) \cong k[x],\quad
				      k[x,y,z]/(y+x, z+x^3) \cong k[x].
			      \]
			      These are integral domains.
			      Also, none is contained in the union of the others.
			      For example,
			      \[
				      (0,0,1) \in V(x,y),\qquad (0,0,1) \notin C_1\cup C_2.
			      \]
			      And when $\operatorname{char} k \neq 2$,
			      \[
				      (1,1,1) \in C_1,\qquad (1,1,1) \notin V(x,y)\cup C_2.
			      \]

		      If $\operatorname{char} k = 2$: then $y + x = y - x$, so $C_1 = C_2$ and
		      \[
			      V(f,g) = V(x,y) \cup V(y - x,\, z - x^3) \quad \text{(two components)}.
		      \]
			      If $\operatorname{char} k \neq 2$:
			      \[
				      V(f,g) = V(x,y) \cup V(y-x,\, z-x^3) \cup V(y+x,\, z+x^3) \quad \text{(three components)}.
			      \]
			      We are using explicitly:
			      \[
				      \text{[Definition, \S 2.1, p.3, Algebraic Geometry, 2026]: } V(S)=\{x\in\mathbb{A}^n_k: f(x)=0\ \forall f\in S\},
			      \]
			      and
			      \[
				      \text{[Lemma 2.1, \S 2.2, p.4, Algebraic Geometry, 2026]: } V(I)\cup V(J)=V(IJ).
			      \]
			      Hence each piece above is an affine subvariety (a zero-locus), and their finite union is again Zariski closed.

			      \subqdash{}

		      \textbf{$V(f,h)$:}\quad
		      On $V(y-x)$: $h = z^2 - x^4 = (z - x^2)(z + x^2)$, giving components $V(y-x, z-x^2)$ and $V(y-x, z+x^2)$.

			      On $V(y+x)$:
			      \[
				      h = z^2 + x^4.
			      \]
			      Since $k$ is algebraically closed, choose $i \in k$ with $i^2 = -1$. Then
			      \[
				      z^2 + x^4 = (z - ix^2)(z + ix^2),
			      \]
			      giving components $V(y+x, z-ix^2)$ and $V(y+x, z+ix^2)$.

		      Each quotient $k[x,y,z]/(y \pm x, z - \alpha x^2) \cong k[x]$ is a domain, so each component is irreducible.

		      If $\operatorname{char} k = 2$: $i = 1$, $y + x = y - x$, and $z + x^2 = z - x^2$, so all four collapse to a single component:
		      \[
			      V(f,h) = V(y - x,\, z - x^2) \quad \text{(one component)}.
		      \]
		      If $\operatorname{char} k \neq 2$: four distinct irreducible components,
		      \[
			      V(f,h) = V(y-x, z-x^2) \cup V(y-x, z+x^2) \cup V(y+x, z-ix^2) \cup V(y+x, z+ix^2).
		      \]

			      \subqdash{}

		      \textbf{$V(f,g,h)$:}\quad
		      We intersect the components of $V(f,g)$ with the condition $h = 0$.

		      On $V(x,y)$: $h = z^2 = 0$, so $z = 0$. This gives the origin $(0,0,0)$.

		      On $C_1 = \{(t,t,t^3)\}$: $h = t^6 - t^4 = t^4(t^2 - 1) = 0$, so $t = 0$ or $t = \pm 1$. Points: $(0,0,0)$, $(1,1,1)$, $(-1,-1,-1)$.

			      On $C_2 = \{(t,-t,-t^3)\}$ (when $\operatorname{char} k \neq 2$):
			      \[
				      h = t^6 + t^4 = t^4(t^2 + 1)=0,
			      \]
			      so $t=0$ or $t=\pm i$.
			      Hence the points are
			      \[
				      (0,0,0),\ (i, -i, i),\ (-i, i, -i).
			      \]

		      If $\operatorname{char} k \neq 2$: $V(f,g,h)$ consists of five points,
		      \[
			      V(f,g,h) = \{(0,0,0),\ (1,1,1),\ (-1,-1,-1),\ (i,-i,i),\ (-i,i,-i)\}.
		      \]
		      If $\operatorname{char} k = 2$: $t^2 - 1 = {(t+1)}^2$, so $t = 0$ or $t = 1$, giving
		      \[
			      V(f,g,h) = \{(0,0,0),\ (1,1,1)\}.
		      \]
		      In both cases, the irreducible components are the individual points.
	      \end{solution}

	\item Let $f, g \in k[x,y]$ be two irreducible polynomials which are not multiples of each other.
	      \begin{enumerate}
			      \item Suppose that at least one of $f$ and $g$ contains a nonzero term in $y$
			            (i.e.\ is not an element of $k[x]$).
			            Use Gauss's Lemma to show that $f, g$ have no common factors in $k(x)[y]$.
		            \begin{solution}
			            View $f, g$ as elements of $R[y]$ where $R = k[x]$ is a UFD with fraction field $K = k(x)$.

				            Suppose first that $f$ involves $y$, i.e.\ $\deg_y f \geq 1$.
				            Then $f$ is primitive in $R[y]$:
				            if a non-unit $c(x)\in R$ divided every coefficient of $f$ (viewed in $y$),
				            then
				            \[
					            f = c(x)\,\tilde f(x,y),
				            \]
				            with $\tilde f$ still involving $y$, contradicting irreducibility of $f$ in $k[x,y]$.
				            Hence, by Gauss's Lemma, $f$ is irreducible in $K[y]$.

				            If $g$ also involves $y$, the same argument shows $g$ is irreducible in $K[y]$.
				            Since $f$ and $g$ are not multiples in $k[x,y]$, they are not associates in $K[y]$.
				            Indeed, if
				            \[
					            f = (a/b)\,g,\qquad a,b\in R\setminus\{0\},
				            \]
				            then $bf=ag$.
				            Since $f$ is irreducible in $R[y]$ with $\deg_y f\ge 1$, no non-unit $a\in R$ divides $f$.
				            So $a$ is a unit; similarly $b$ is a unit.
				            This would force $f,g$ to be scalar multiples in $k[x,y]$, contradiction.
				            Thus $f,g$ are distinct irreducibles in the PID $K[y]$, hence coprime.

			            If $g \in k[x]$: then $g$ is a nonzero element of $R \subset K$, hence a unit in $K[y]$. So $\gcd(f, g) = 1$ trivially.
			            \end{solution}

			      \subqdash{}

		      \item Show that there exist nonzero polynomials $h \in k[x]$ and $p, q \in k[x,y]$ such that $h = fp + gq$.
		            \begin{solution}
			            By part (a), $f$ and $g$ are coprime in the PID $k(x)[y]$. By the Euclidean algorithm, there exist $P, Q \in k(x)[y]$ with
			            \[
				            1 = fP + gQ.
			            \]
				            Write
				            \[
					            P = p_0/d_1,\qquad Q = q_0/d_2,
				            \]
				            where $p_0,q_0\in k[x,y]$ and $d_1,d_2\in k[x]\setminus\{0\}$.
				            Multiplying through by $d_1d_2$:
			            \[
				            d_1 d_2 = f(d_2 p_0) + g(d_1 q_0).
			            \]
			            Setting $h = d_1 d_2 \in k[x] \setminus \{0\}$, $p = d_2 p_0$, $q = d_1 q_0$ gives the result.
		            \end{solution}

			      \subqdash{}

		      \item Show that the set $\{x : (x,y) \in V(f,g)\}$ of first coordinates of points of $V(f,g)$ is finite.
		            \begin{solution}
			            By part (b), $h(x) = f(x,y)\, p(x,y) + g(x,y)\, q(x,y)$ for some nonzero $h \in k[x]$.

			            If $(a, b) \in V(f, g)$, then $f(a, b) = g(a, b) = 0$, so
			            \[
				            h(a) = f(a,b)\, p(a,b) + g(a,b)\, q(a,b) = 0.
			            \]
			            Since $h \in k[x]$ is nonzero, it has at most $\deg h$ roots. Therefore the set of first coordinates is finite.
		            \end{solution}

			      \subqdash{}

		      \item Show that the set $V(f,g)$ is finite.
		            \begin{solution}
				            By part (c), the set
				            \[
					            A=\{a\in k : (a,b)\in V(f,g)\text{ for some }b\}
				            \]
				            is finite.
				            For each $a\in A$, the fiber
				            \[
					            \{b\in k : (a,b)\in V(f,g)\}
				            \]
				            is contained in
				            \[
					            \{b\in k : f(a,b)=0\}.
				            \]

				            Since at least one of $f,g$ involves $y$, say $f$, write
				            \[
					            f=\sum_j c_j(x)y^j,\qquad c_m\neq 0,\ m\ge 1.
				            \]
				            If $f(a,y)\equiv 0$ as a polynomial in $y$, then $c_j(a)=0$ for all $j$.
				            So $(x-a)$ divides every $c_j(x)$, hence $(x-a)\mid f$ in $k[x,y]$.
				            But $f$ is irreducible with $\deg_y f\ge 1$, contradiction.
				            Therefore $f(a,y)$ is nonzero for every $a$, so each fiber is finite.

			            A finite union of finite sets is finite, so $V(f, g)$ is finite.
		            \end{solution}
	      \end{enumerate}

		\item Let $n,m\ge 1$ and consider
		      \[
			      \varphi:\mathbb{A}^1\to\mathbb{A}^2,\qquad t\mapsto (t^n,t^m).
		      \]
		      Show that $\operatorname{im}(\varphi)$ is an affine subvariety of $\mathbb{A}^2$.
		      Give the condition for which $\varphi$ is bijective onto its image.
		      In that case, give a birational inverse when $\operatorname{char}k=0$.
	      \begin{solution}
		      Let $d = \gcd(n, m)$, $a = n/d$, $b = m/d$, so $\gcd(a, b) = 1$.

			      \textbf{Image is a subvariety:}\quad
			      By [Definition, \S 2.1, p.3, Algebraic Geometry, 2026],
			      \[
				      \text{it is enough to prove } \operatorname{im}(\varphi)=V(S)\text{ for some }S\subset k[x,y].
			      \]
			      Since $k$ is algebraically closed, the map $t \mapsto t^d$ is surjective on $k$, so
			      \[
				      \operatorname{im}(\varphi) = \{(t^n, t^m) : t \in k\} = \{(s^a, s^b) : s \in k\}.
			      \]

		      \begin{claim}
			      $\operatorname{im}(\varphi) = V(x^b - y^a)$.
		      \end{claim}

		      If $(x, y) = (s^a, s^b)$, then $x^b - y^a = s^{ab} - s^{ab} = 0$, so $\operatorname{im}(\varphi) \subseteq V(x^b - y^a)$.

			      Conversely, let $(x, y) \in V(x^b - y^a)$.
			      If $x = 0$, then $y^a = 0$, so $y = 0 = (0^a,0^b)\in \operatorname{im}(\varphi)$.
			      If $x\neq 0$, choose $s_0\in k$ with $s_0^a=x$.
			      Then $s_0^{ab}=x^b=y^a$, so
			      \[
				      {\left(\frac{y}{s_0^b}\right)}^a = 1.
			      \]
			      Let $\zeta := y/s_0^b \in \mu_a$.
			      Since $\gcd(a,b)=1$, the map
			      \[
				      \mu_a\to\mu_a,\qquad \eta\mapsto\eta^b
			      \]
			      is bijective.
			      Hence there exists $\eta\in\mu_a$ with $\eta^b=\zeta$.
			      Set $s:=\eta s_0$. Then
			      \[
				      s^a = \eta^a s_0^a = x,\qquad s^b = \eta^b s_0^b = \zeta s_0^b = y.
			      \]
			      Thus $(x, y) = (s^a, s^b) \in \operatorname{im}(\varphi)$.

			      The polynomial $x^b-y^a$ is irreducible.
			      Consider
			      \[
				      k[x,y]\to k[t],\qquad x\mapsto t^a,\ y\mapsto t^b.
			      \]
			      Its image is $k[t^a,t^b]$.
				      The monomials $\{x^{i}y^{j}: i\geq 0,\ 0\leq j\leq a-1\}$ map to $\{t^{ia+jb}\}$,
			      and these exponents are distinct since $\gcd(a,b)=1$.
			      Hence
			      \[
				      k[x,y]/(x^b-y^a)\hookrightarrow k[t]
			      \]
			      is a domain, so $x^b-y^a$ is irreducible.

			      Therefore
				      \[
					      \operatorname{im}(\varphi) = V(x^b-y^a),
				      \]
				      so the image is an affine subvariety in the sense of [Definition, \S 2.1, p.3, Algebraic Geometry, 2026].

			      \textbf{Bijectivity:}\quad
			      $\varphi$ is bijective onto its image iff
			      \[
				      t^n=s^n\ \text{and}\ t^m=s^m \ \Longrightarrow\ t=s.
			      \]
			      For $t,s\neq 0$, set $\omega=s/t$.
			      Then
			      \[
				      \omega^n=\omega^m=1 \Longrightarrow \omega^d=1.
			      \]
			      So we need the only such $\omega$ to be $1$.

		      In characteristic zero, $\mu_d(k) = \{1\}$ iff $d = 1$. So $\varphi$ is bijective to its image if and only if $\gcd(n, m) = 1$.

			      In characteristic $p>0$, $\varphi$ is bijective iff $\gcd(n,m)$ is a power of $p$,
			      since the only $p^e$-th root of unity in $k$ is $1$.

			      \textbf{Birational inverse (char $0$, $\gcd(n,m) = 1$):}\quad
			      Here we invoke the Chapter 7 framework:
			      \[
				      \text{[Definition, \S 7, p.22, Algebraic Geometry, 2026]: rational maps/functions and }k(V),
			      \]
			      \[
				      \text{[Definition, \S 7, p.25, Algebraic Geometry, 2026]: birational equivalence via rational inverses.}
			      \]
			      By B\'{e}zout, choose $\alpha, \beta \in \mathbb{Z}$ with $\alpha n + \beta m = 1$. Define the rational map
			      \[
				      \psi \colon V(x^m - y^n) \dashrightarrow \mathbb{A}^1, \qquad \psi(x, y) = x^\alpha y^\beta,
			      \]
		      (where negative exponents denote division). Then
		      \[
			      \psi(\varphi(t)) = {(t^n)}^\alpha {(t^m)}^\beta = t^{\alpha n + \beta m} = t,
		      \]
			      On the dense open subset where $x \neq 0$ and $y \neq 0$, this is a composition of regular functions. In function fields,
			      \[
				      \psi^*(t) = x^\alpha y^\beta,
				      \qquad
				      \varphi^*(x) = t^n,\ \varphi^*(y)=t^m,
			      \]
			      so
			      \[
				      (\varphi^* \circ \psi^*)(t) = {(t^n)}^\alpha {(t^m)}^\beta = t^{\alpha n + \beta m} = t.
			      \]
			      Also
			      \[
				      (\psi^* \circ \varphi^*)(x)=x,\qquad (\psi^* \circ \varphi^*)(y)=y,
			      \]
			      so
			      \[
				      \psi^* \circ \varphi^*=\mathrm{id}_{k(V(x^m-y^n))}.
			      \]
				      Hence $\psi$ and $\varphi$ are inverse as rational maps ([Definition, \S 7, p.25, Algebraic Geometry, 2026]).
				      So $\psi$ is a birational inverse of $\varphi$.
		      \end{solution}

	\item
	      \begin{enumerate}
			      \item Let $S=V(x^2+y^2-1)$ be the circle and $H=V(wz-1)$ be the hyperbola.
			            Show that either $S\cong H$ or $S\cong\mathbb{A}^1$.
		            \begin{solution}
			            \textbf{Case $\operatorname{char} k \neq 2$:}\quad
			            Since $k$ is algebraically closed, choose $i \in k$ with $i^2 = -1$. Define
			            \[
				            \Phi \colon S \to H, \qquad (x, y) \mapsto (x + iy,\, x - iy).
			            \]
			            On $S$: $(x + iy)(x - iy) = x^2 + y^2 = 1$, so $\Phi$ maps into $H$. The inverse is
			            \[
				            \Phi^{-1} \colon H \to S, \qquad (w, z) \mapsto \left(\frac{w + z}{2},\, \frac{w - z}{2i}\right),
			            \]
				            which is well-defined since $\operatorname{char} k \neq 2$.
				            One verifies
				            \[
					            \Phi^{-1}\circ\Phi=\mathrm{id}_S,\qquad
					            \Phi\circ\Phi^{-1}=\mathrm{id}_H.
				            \]
				            Both maps are morphisms, so $S \cong H$.

			            \textbf{Case $\operatorname{char} k = 2$:}\quad
				            In characteristic $2$,
				            \[
					            x^2+y^2-1={(x+y)}^2+1={(x+y+1)}^2,
				            \]
				            since ${(x+y+1)}^2=x^2+y^2+1$ and $-1=1$.
				            Therefore
			            \[
				            S = V({(x + y + 1)}^2) = V(x + y + 1),
			            \]
				            which is a line in $\mathbb{A}^2$.
				            The map
				            \[
					            t\mapsto (t,t+1)
				            \]
				            is an isomorphism $\mathbb{A}^1\xrightarrow{\sim}S$ with inverse $(x,y)\mapsto x$.
				            So $S\cong\mathbb{A}^1$.
			            \end{solution}

			      \subqdash{}

			      \item Let $k$ have characteristic $p$.
			            Show that
			            \[
				            \varphi:\mathbb{A}^1\to\mathbb{A}^1,\qquad t\mapsto t^p
			            \]
			            is a bijection, and show that $\varphi$ is not a birational equivalence.
			            \begin{solution}
				            \textbf{Bijection:}

				            \emph{Surjectivity:}
				            For any $a\in k$, the polynomial $t^p-a$ has a root in $k$,
				            since $k$ is algebraically closed.

				            \emph{Injectivity:}
				            If $s^p=t^p$, then
				            \[
					            {(s-t)}^p=s^p-t^p=0,
				            \]
				            because $\binom{p}{i}=0$ in characteristic $p$ for $0<i<p$.
				            Hence $s=t$.

			      \textbf{Not a birational equivalence:}\quad
			      By [Corollary 7.3, \S 7, p.26, Algebraic Geometry, 2026],
			      irreducible affine varieties are birational iff their function fields are $k$-isomorphic.
			      Here
			      \[
				      \varphi^* \colon k(t)\to k(t),\qquad f(t)\mapsto f(t^p),
			      \]
			      and
			      \[
				      \operatorname{im}(\varphi^*)=k(t^p)\subsetneq k(t).
			      \]
			      Moreover
			      \[
				      [k(t):k(t^p)]=p>1,
			      \]
			      because $t$ is algebraic over $k(t^p)$ with equation
			      \[
				      X^p-t^p=0.
			      \]
			      Therefore $\varphi^*$ is not an isomorphism, hence $\varphi$ is not birational.
		            \end{solution}
	      \end{enumerate}

	\item Consider the cubic curve $C := V(y^2 - x^3 - x) \subseteq \mathbb{A}^2$.
	      \begin{enumerate}
		      \item Prove that $C$ is irreducible.
		            \begin{solution}
				            View $F=y^2-x^3-x$ as an element of $k[x][y]$.
				            If $F$ were reducible in $k[x,y]$, then since $\deg_y F=2$, it would factor as
			            \[
				            F = (y - p(x))(y - q(x))
			            \]
				            for some $p,q\in k[x]$, giving
				            \[
					            p+q=0,\qquad pq=-(x^3+x).
				            \]
				            Thus $q=-p$ and $p^2=x^3+x$.
				            But $\deg(p^2)=2\deg p$ is even, while $\deg(x^3+x)=3$ is odd.
				            Contradiction. Therefore $F$ is irreducible.
			            \end{solution}

			      \subqdash{}

		      \item Find the domain of definition of the rational map
		            \[
			            \varphi \colon C \dashrightarrow \mathbb{A}^1,\qquad \varphi(x,y)=x/y.
		            \]
			            \begin{solution}
			            On $C$, $y^2 = x^3 + x = x(x^2 + 1)$, so
			            \[
				            \frac{x}{y} = \frac{x}{y} \cdot \frac{y}{y} = \frac{xy}{y^2} = \frac{xy}{x(x^2 + 1)} = \frac{y}{x^2 + 1}.
			            \]
				            The representation $x/y$ is regular where $y\neq 0$.
				            The representation $y/(x^2+1)$ is regular where $x^2+1\neq 0$.
				            Together, $\varphi$ is regular on
				            \[
					            C\setminus V(y,x^2+1).
				            \]

				            At a point $P\in C$ with $y{(P)}=0$ and ${x(P)}^2+1=0$:
				            \[
					            {x(P)}^2=-1,\qquad y{(P)}=0.
				            \]
				            Locally, $y^2=x(x^2+1)$ has a simple zero in $(x-x(P))$ at such $P$.
				            Set $u=x-x(P)$.
				            Then
				            \[
					            y^2\sim x(P)\cdot u\cdot(\text{unit}),
				            \]
				            so $y\sim \sqrt{u}$.
				            Hence
				            \[
					            \varphi=\frac{y}{x^2+1}\sim \frac{\sqrt{u}}{u}\to\infty.
				            \]
				            So $\varphi$ has a pole at $P$ and cannot be extended.

			            Therefore the domain of definition is $C \setminus V(y,\, x^2 + 1)$.

				            Explicitly, the excluded points satisfy $x^2=-1$ and $y=0$:
				            \[
					            (i,0),\ (-i,0)\ \text{if }\operatorname{char}k\neq 2;\qquad (1,0)\ \text{if }\operatorname{char}k=2.
				            \]
			            \end{solution}

			      \subqdash{}

			      \item Now consider $C' := V(y^2 - x^3 - x^2)\subseteq\mathbb{A}^2$.
			            The same formula defines a rational map $\varphi:C'\dashrightarrow\mathbb{A}^1$.
			            Find a dominant rational map $\psi:\mathbb{A}^1\dashrightarrow C'$ with
			            \[
				            \varphi\psi=\mathrm{id}_{\mathbb{A}^1}.
			            \]
			            \begin{solution}
			            We seek $\psi(t) = (a(t), b(t)) \in C'$ with $\varphi(\psi(t)) = a/b = t$, i.e.\ $a = tb$.

			            Substituting into $b^2 = a^3 + a^2 = {(tb)}^3 + {(tb)}^2$:
			            \[
				            b^2 = t^3 b^3 + t^2 b^2, \qquad b^2(1 - t^2) = t^3 b^3, \qquad b = \frac{1 - t^2}{t^3},
			            \]
			            and then $a = tb = \dfrac{1 - t^2}{t^2}$.

			            Define the rational map
			            \[
				            \psi \colon \mathbb{A}^1 \dashrightarrow C', \qquad t \mapsto \left(\frac{1 - t^2}{t^2},\, \frac{1 - t^2}{t^3}\right).
			            \]

				            \emph{Verification}: $\varphi(\psi(t)) = \dfrac{(1-t^2)/t^2}{(1-t^2)/t^3} = t$.\,\checkmark{}
				            $\psi(t)\in C'$:
				            \begin{align*}
					            y^2-x^3-x^2
					            &= \frac{{(1-t^2)}^2}{t^6}
					            - \frac{{(1-t^2)}^3}{t^6}
					            - \frac{{(1-t^2)}^2}{t^4} \\
					            &= \frac{{(1-t^2)}^2\bigl(1-(1-t^2)-t^2\bigr)}{t^6} \\
					            &= 0.
				            \end{align*}
				            \checkmark{}


				            \emph{Dominance:}
				            $\psi$ is defined for $t\neq 0$.
				            Every $(x,y)\in C'$ with $y\neq 0$ is in the image:
				            set $t=x/y$, then recover $\psi(t)=(x,y)$ on $C'$.
				            Since $C'\setminus V(y)$ is dense in irreducible $C'$, $\psi$ is dominant.
		            \end{solution}
	      \end{enumerate}
\end{enumerate}

\end{document}
