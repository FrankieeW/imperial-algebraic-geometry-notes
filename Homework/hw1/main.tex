% Copyright (c) 2026 Frankie Feng-Cheng WANG. All rights reserved.
% Repository: https://github.com/FrankieeW/
% !TEX program = xelatex



%----------------------------------------------------------------------------------------
% PACKAGES AND OTHER DOCUMENT CONFIGURATIONS
%----------------------------------------------------------------------------------------

\documentclass[
  12pt, % Default font size, values between 10pt-12pt are allowed
]{assignment}

% Set table of contents depth to section level only (no subsection, subsubsection)
\setcounter{tocdepth}{1}
\setcounter{secnumdepth}{2} % Still number sections up to subsubsection, just don't show in TOC

\counterwithin{equation}{section}
\counterwithin{figure}{section}

\usepackage{amsthm} % For proof environment
\newtheorem{lemma}{Lemma}
\newtheorem*{claim}{Claim}
\newtheorem*{corollary}{Corollary}
\newtheorem*{remark}{Remark}
\newtheorem*{definition}{Definition}
\newtheorem*{example}{Example}
\newenvironment{solution}
{\par\noindent\rule{\linewidth}{0.4pt}\par\smallskip\begin{proof}[Solution]}
{\end{proof}\par\noindent\rule{\linewidth}{0.4pt}\par\smallskip}

\usepackage{float}
\usepackage{hyperref}
\usepackage{mathtools}
\usepackage{amssymb}
\usepackage{xcolor}
\usepackage{mathrsfs}
\usepackage{enumitem}
\usepackage{tikz}
\usetikzlibrary{arrows,shapes.geometric,calc}

\newcommand\nb{\addtocounter{equation}{1}\tag{\theequation}}
\newcommand{\subqdash}{%
  \par\vspace{0.8\baselineskip}%
  \noindent\makebox[\linewidth][l]{\leaders\hbox{-}\hfill\kern0pt}%
  \par\vspace{0.8\baselineskip}%
}
\pgfplotsset{compat=1.18}
\renewcommand{\thesection}{Exercise \arabic{section}}

%-------------------------------------------------------------------------------
% ASSIGNMENT INFORMATION
%-------------------------------------------------------------------------------

% Fix for fancyhdr warning: increase headheight and adjust topmargin
\setlength{\headheight}{29.35158pt}
\addtolength{\topmargin}{-17.35158pt}

\title{Assessed Coursework 1} % Assignment title
\author{Frankie Feng-Cheng WANG} % Student name
\email{maths@frankie.wang\\fw225@ic.ac.uk} % Student email
\date{16th February 2026, 1 PM} % Due date
\institute{Department of Mathematics\\Imperial College London} % Institute or school name
\course{MATH70029-Algebraic Geometry} % Course or class name
\lecturer{Matt Booth} % Lecturer or teacher in charge of the assignment

%----------------------------------------------------------------------------------------
\begin{document}
\setcounter{tocdepth}{3}
\maketitle % Output the assignment title, created automatically using the information in the custom commands above
% \tableofcontents

%----------------------------------------------------------------------------------------
% ASSIGNMENT CONTENT
%----------------------------------------------------------------------------------------
50 points: 10 + 10 + 10 + 10 + 10.\\
As always, we let $k$ be an algebraically closed field. Some answers may depend on the characteristic of $k$.
\section{}
Consider the following polynomials in $k[x,y,z]$:
	      \[
		      f = y^2 - x^2, \qquad g = x^4 - yz, \qquad h = z^2 - x^3 y.
	      \]
	      Find and describe the irreducible components of the varieties $V(f,g)$, $V(f,h)$, $V(f,g,h)$.
	      \begin{solution}
		      Throughout, note that $f = (y-x)(y+x)$.

			      \textbf{$V(f,g)$:}\quad
			      On $V(y-x)$: substituting $y = x$ into $g$ gives $x^4 - xz = x(x^3 - z) = 0$, so $x = 0$ or $z = x^3$.
		      \begin{itemize}
			      \item $x = 0, y = 0$: the $z$-axis $V(x,y) = \{(0,0,t) : t \in k\}$.
			      \item $z = x^3, y = x$: the twisted cubic $C_1 = V(y - x,\, z - x^3) = \{(t, t, t^3) : t \in k\}$.
		      \end{itemize}
		      On $V(y+x)$: substituting $y = -x$ gives $x^4 + xz = x(x^3 + z) = 0$, so $x = 0$ (the $z$-axis again) or $z = -x^3$.
		      \begin{itemize}
			      \item $z = -x^3, y = -x$: the curve $C_2 = V(y + x,\, z + x^3) = \{(t, -t, -t^3) : t \in k\}$.
		      \end{itemize}
			      Each component is irreducible:
			      \[
				      k[x,y,z]/(x,y) \cong k[z],\quad
				      k[x,y,z]/(y-x, z-x^3) \cong k[x],\quad
				      k[x,y,z]/(y+x, z+x^3) \cong k[x].
			      \]
			      These are integral domains.
			      Also, none is contained in the union of the others.
			      For example,
			      \[
				      (0,0,1) \in V(x,y),\qquad (0,0,1) \notin C_1\cup C_2.
			      \]
			      And when $\operatorname{char} k \neq 2$,
			      \[
				      (1,1,1) \in C_1,\qquad (1,1,1) \notin V(x,y)\cup C_2.
			      \]
			      Also, when $\operatorname{char} k \neq 2$,
			      \[
				      (1,-1,-1)\in C_2,\qquad (1,-1,-1)\notin V(x,y)\cup C_1.
			      \]

		      If $\operatorname{char} k = 2$: then $y + x = y - x$, so $C_1 = C_2$ and
		      \[
			      V(f,g) = V(x,y) \cup V(y - x,\, z - x^3) \quad \text{(two components)}.
		      \]
			      Here
			      \[
				      (0,0,1)\in V(x,y)\setminus V(y-x,z-x^3),\qquad (1,1,1)\in V(y-x,z-x^3)\setminus V(x,y),
			      \]
			      so neither component contains the other.
			      If $\operatorname{char} k \neq 2$:
			      \[
				      V(f,g) = V(x,y) \cup V(y-x,\, z-x^3) \cup V(y+x,\, z+x^3) \quad \text{(three components)}.
			      \]
			      We are using explicitly:
			      \[
				      \text{[Definition, \S 2.1, p.3, Algebraic Geometry, 2026]: } V(S)=\{x\in\mathbb{A}^n_k: f(x)=0\ \forall f\in S\},
			      \]
			      and
			      \[
				      \text{[Lemma 2.1, \S 2.2, p.4, Algebraic Geometry, 2026]: } V(I)\cup V(J)=V(IJ).
			      \]
			      Hence each piece above is an affine subvariety (a zero-locus), and their finite union is again Zariski closed.

			      \subqdash{}

		      \textbf{$V(f,h)$:}\quad
		      On $V(y-x)$: $h = z^2 - x^4 = (z - x^2)(z + x^2)$, giving components $V(y-x, z-x^2)$ and $V(y-x, z+x^2)$.

			      On $V(y+x)$:
			      \[
				      h = z^2 + x^4.
			      \]
			      Since $k$ is algebraically closed, choose $i \in k$ with $i^2 = -1$. Then
			      \[
				      z^2 + x^4 = (z - ix^2)(z + ix^2),
			      \]
			      giving components $V(y+x, z-ix^2)$ and $V(y+x, z+ix^2)$.

		      Each quotient $k[x,y,z]/(y \pm x, z - \alpha x^2) \cong k[x]$ is a domain, so each component is irreducible.

		      If $\operatorname{char} k = 2$: $i = 1$, $y + x = y - x$, and $z + x^2 = z - x^2$, so all four collapse to a single component:
		      \[
			      V(f,h) = V(y - x,\, z - x^2) \quad \text{(one component)}.
		      \]
		      If $\operatorname{char} k \neq 2$: four distinct irreducible components,
		      \[
			      V(f,h) = V(y-x, z-x^2) \cup V(y-x, z+x^2) \cup V(y+x, z-ix^2) \cup V(y+x, z+ix^2).
		      \]

			      \subqdash{}

		      \textbf{$V(f,g,h)$:}\quad
		      We intersect the components of $V(f,g)$ with the condition $h = 0$.

		      On $V(x,y)$: $h = z^2 = 0$, so $z = 0$. This gives the origin $(0,0,0)$.

		      On $C_1 = \{(t,t,t^3)\}$: $h = t^6 - t^4 = t^4(t^2 - 1) = 0$, so $t = 0$ or $t = \pm 1$. Points: $(0,0,0)$, $(1,1,1)$, $(-1,-1,-1)$.

			      On $C_2 = \{(t,-t,-t^3)\}$ (when $\operatorname{char} k \neq 2$):
			      \[
				      h = t^6 + t^4 = t^4(t^2 + 1)=0,
			      \]
			      so $t=0$ or $t=\pm i$.
			      Hence the points are
			      \[
				      (0,0,0),\ (i, -i, i),\ (-i, i, -i).
			      \]

		      If $\operatorname{char} k \neq 2$: $V(f,g,h)$ consists of five points,
		      \[
			      V(f,g,h) = \{(0,0,0),\ (1,1,1),\ (-1,-1,-1),\ (i,-i,i),\ (-i,i,-i)\}.
		      \]
		      If $\operatorname{char} k = 2$: $t^2 - 1 = {(t+1)}^2$, so $t = 0$ or $t = 1$, giving
		      \[
			      V(f,g,h) = \{(0,0,0),\ (1,1,1)\}.
		      \]
		      In both cases, the irreducible components are the individual points.
	      \end{solution}

\section{}
Let $f, g \in k[x,y]$ be two irreducible polynomials which are not multiples of each other.
	      \begin{enumerate}[label=(\alph*)]
			      \item Suppose that at least one of $f$ and $g$ contains a nonzero term in $y$
			            (i.e.\ is not an element of $k[x]$).
			            Use Gauss's Lemma to show that $f, g$ have no common factors in $k(x)[y]$.
		            \begin{solution}
			            Let
			            \[
				            R:=k[x],\qquad K:=k(x),\qquad f,g\in R[y].
			            \]

			            Assume $\deg_y f\ge 1$.
			            If $f$ were not primitive, then
			            \[
				            f=c(x)\tilde f(x,y),
			            \]
			            with non-unit $c(x)\in R$, contradiction.
			            So $f$ is primitive, hence Gauss implies
			            \[
				            f\ \text{irreducible in }K[y].
			            \]

			            If $\deg_y g\ge 1$, the same gives $g$ irreducible in $K[y]$.
			            If $f,g$ were associates in $K[y]$,
			            \[
				            f=(a/b)g,\qquad a,b\in R\setminus\{0\},\ \gcd(a,b)=1,
			            \]
			            (after cancelling common factors in $R$), so
			            \[
				            bf=ag.
			            \]
			            Since $a\mid ag=bf$ and $\gcd(a,b)=1$, Euclid's Lemma in the UFD $R[y]$ gives $a\mid f$.
			            Since $\deg_y f\ge 1$ and $f$ irreducible, $a\in k^\times$; similarly $b\in k^\times$.
			            Hence $f,g$ are scalar multiples in $k[x,y]$, contradiction.
			            Therefore
			            \[
				            \gcd_{K[y]}(f,g)=1.
			            \]

			            If $g\in k[x]\subset K$, then $g$ is a unit of $K[y]$, so again
			            \[
				            \gcd_{K[y]}(f,g)=1.
			            \]
		            \end{solution}

			      \subqdash{}

		      \item Show that there exist nonzero polynomials $h \in k[x]$ and $p, q \in k[x,y]$ such that $h = fp + gq$.
		            \begin{solution}
			            From (a),
			            \[
				            \gcd_{k(x)[y]}(f,g)=1.
			            \]
			            Hence in the PID $k(x)[y]$,
			            \[
				            1=fP+gQ,\qquad P,Q\in k(x)[y].
			            \]
			            Write
			            \[
				            P=\frac{p_0}{d_1},\qquad
				            Q=\frac{q_0}{d_2},
			            \]
			            with
			            \[
				            p_0,q_0\in k[x,y],\qquad d_1,d_2\in k[x]\setminus\{0\}.
			            \]
			            Multiply by $d_1d_2$:
			            \[
				            d_1d_2=f(d_2p_0)+g(d_1q_0).
			            \]
			            Set
			            \[
				            h:=d_1d_2,\qquad p:=d_2p_0,\qquad q:=d_1q_0.
			            \]
			            Then
			            \[
				            h=fp+gq,\qquad h\in k[x]\setminus\{0\}.
			            \]
		            \end{solution}

			      \subqdash{}

		      \item Show that the set $\{x : (x,y) \in V(f,g)\}$ of first coordinates of points of $V(f,g)$ is finite.
		            \begin{solution}
			            By (b), there are
			            \[
				            h\in k[x]\setminus\{0\},\qquad p,q\in k[x,y],\qquad h=fp+gq.
			            \]
			            If $(a,b)\in V(f,g)$, then
			            \[
				            f(a,b)=g(a,b)=0
			            \]
			            and therefore
			            \[
				            h(a)=f(a,b)p(a,b)+g(a,b)q(a,b)=0.
			            \]
			            So
			            \[
				            \{a\in k:\exists b,\ (a,b)\in V(f,g)\}\subseteq V(h),
			            \]
			            hence finite.
		            \end{solution}

			      \subqdash{}

		      \item Show that the set $V(f,g)$ is finite.
		            \begin{solution}
			            Let
			            \[
				            A:=\{a\in k:\exists b,\ (a,b)\in V(f,g)\}.
			            \]
			            By (c), $A$ is finite.

			            For each $a\in A$, let
			            \[
				            F_a:=\{b\in k:(a,b)\in V(f,g)\}.
			            \]
			            Then
			            \[
				            F_a\subseteq\{b\in k:f(a,b)=0\}.
			            \]

			            w.l.o.g., assume $\deg_y f\ge 1$.\footnote{By part (a), at least one of $f,g$ has positive $y$-degree. If instead $\deg_y f=0$, then $\deg_y g\ge 1$, and the same argument below applies after swapping $f$ and $g$. This is valid since $V(f,g)=V(g,f)$.}
			            Write
			            \[
				            f=\sum_{j=0}^m c_j(x)y^j,\qquad c_m\neq 0,\ m\ge 1.
			            \]
			            If $f(a,y)\equiv 0$, then $c_j(a)=0$ for all $j$, so
			            \[
				            (x-a)\mid c_j(x)\ \forall j
				            \Longrightarrow
				            (x-a)\mid f,
			            \]
			            contradiction (irreducibility of $f$ and $\deg_y f\ge 1$).

			            Hence $f(a,y)\neq 0$, so each $F_a$ is finite.
			            Finally,
			            \[
				            V(f,g)=\bigcup_{a\in A}\{a\}\times F_a
			            \]
			            is finite.
		            \end{solution}
	      \end{enumerate}

\section{}
Let $n,m\ge 1$ and consider
	      \[
		      \varphi:\mathbb{A}^1\to\mathbb{A}^2,\qquad t\mapsto (t^n,t^m).
	      \]
		      Show that $\operatorname{im}(\varphi)$ is an affine subvariety of $\mathbb{A}^2$.
		      Give the condition for which $\varphi$ is bijective onto its image.
		      In that case, give a birational inverse when $\operatorname{char}k=0$.
	      \begin{solution}
		      Let $d = \gcd(n, m)$, $a = n/d$, $b = m/d$, so $\gcd(a, b) = 1$.

			      \textbf{Image is a subvariety:}\quad
			      By [Definition, \S 2.1, p.3, Algebraic Geometry, 2026],
			      \[
				      \text{it is enough to prove } \operatorname{im}(\varphi)=V(S)\text{ for some }S\subset k[x,y].
			      \]
			      Since $k$ is algebraically closed, the map $t \mapsto t^d$ is surjective on $k$, so
			      \[
				      \operatorname{im}(\varphi) = \{(t^n, t^m) : t \in k\} = \{(s^a, s^b) : s \in k\}.
			      \]

		      \begin{claim}
			      $\operatorname{im}(\varphi) = V(x^b - y^a)$.
		      \end{claim}

		      \[
			      (s^a,s^b)\in\operatorname{im}(\varphi)
			      \Longrightarrow
			      {(s^a)}^b-{(s^b)}^a=0.
		      \]
		      Hence
		      \[
			      \operatorname{im}(\varphi)\subseteq V(x^b-y^a).
		      \]

			      Conversely, let $(x,y)\in V(x^b-y^a)$.
			      If $x=0$, then $y^a=0$, so $(x,y)=(0,0)\in\operatorname{im}(\varphi)$.
			      Assume $x\neq 0$ and choose $s_0\in k$ with $s_0^a=x$. Then
			      \[
				      s_0^{ab}=x^b=y^a,
			      \]
			      so
			      \[
				      {\left(\frac{y}{s_0^b}\right)}^a = 1.
			      \]
			      Let $\zeta := y/s_0^b \in \mu_a$.
			      Since $\gcd(a,b)=1$, the map
			      \[
				      \mu_a\to\mu_a,\qquad \eta\mapsto\eta^b
			      \]
			      is bijective.
			      Hence there exists $\eta\in\mu_a$ with $\eta^b=\zeta$.
			      Set $s:=\eta s_0$. Then
			      \[
				      s^a = \eta^a s_0^a = x,\qquad s^b = \eta^b s_0^b = \zeta s_0^b = y.
			      \]
			      Thus $(x,y)=(s^a,s^b)\in\operatorname{im}(\varphi)$.

				      Consider
				      \[
					      k[x,y]\to k[t],\qquad x\mapsto t^a,\ y\mapsto t^b.
				      \]
				      Let $\theta$ denote this map. Clearly
				      \[
					      (x^b-y^a)\subseteq \ker\theta.
				      \]
				      For $F\in k[x,y]$, reduce modulo $(x^b-y^a)$ to
				      \[
					      F \equiv \sum_{j=0}^{a-1} f_j(x)y^j.
				      \]
				      If $F\in\ker\theta$, then
				      \[
					      0
					      =
					      \sum_{j=0}^{a-1} f_j(t^a)t^{bj}
					      =
					      \sum_{j=0}^{a-1}\sum_{i\ge 0} c_{ij}\,t^{ai+bj}.
				      \]
				      If
				      \[
					      ai+bj=ai'+bj',\qquad 0\le j,j'<a,
				      \]
				      then
				      \[
					      a(i-i')=b(j'-j).
				      \]
				      Since $\gcd(a,b)=1$, we get $a\mid(j'-j)$, hence $j=j'$ and then $i=i'$.\footnote{This uniqueness of $ai+bj$ for $0\le j<a$ is the only arithmetic input.}
				      Therefore all $c_{ij}=0$, so every $f_j=0$, hence $F\in (x^b-y^a)$.
				      Thus
				      \[
					      \ker\theta=(x^b-y^a).
				      \]
				      Hence
				      \[
					      k[x,y]/(x^b-y^a)\cong k[t^a,t^b]\subset k[t]
				      \]
				      is a domain. So $(x^b-y^a)$ is prime, therefore $x^b-y^a$ is irreducible.

			      Therefore
			      \[
				      \operatorname{im}(\varphi)=V(x^b-y^a),
			      \]
			      an affine subvariety by [Definition, \S 2.1, p.3, Algebraic Geometry, 2026].

			      \textbf{Bijectivity:}\quad
			      \[
				      \varphi(t)=\varphi(s)
				      \Longleftrightarrow
				      t^n=s^n,\ t^m=s^m.
			      \]
			      For $t,s\neq 0$, set $\omega=s/t$. Then
			      \[
				      \omega^n=\omega^m=1
				      \Longleftrightarrow
				      \omega^d=1.
			      \]
			      So injectivity is equivalent to
			      \[
				      \mu_d(k)=\{1\}.
			      \]

		      In characteristic zero,
		      \[
			      \mu_d(k)=\{1\}
			      \Longleftrightarrow
			      d=1.
		      \]
		      Hence bijective iff
		      \[
			      \gcd(n,m)=1.
		      \]

			      In characteristic $p>0$,
			      \[
				      \mu_d(k)=\{1\}
				      \Longleftrightarrow
				      d=p^e,
			      \]
			      so bijective iff $\gcd(n,m)$ is a power of $p$.

			      \textbf{Birational inverse (char $0$, $\gcd(n,m) = 1$):}\quad
			      Here we invoke the Chapter 7 framework:
			      \[
				      \text{[Definition, \S 7, p.22, Algebraic Geometry, 2026]: rational maps/functions and }k(V),
			      \]
			      \[
				      \text{[Definition, \S 7, p.25, Algebraic Geometry, 2026]: birational equivalence via rational inverses.}
			      \]
			      Choose $\alpha,\beta\in\mathbb{Z}$ with
			      \[
				      \alpha n+\beta m=1.
			      \]
				      Define
				      \[
					      \psi \colon V(x^b - y^a) \dashrightarrow \mathbb{A}^1, \qquad \psi(x, y) = x^\alpha y^\beta,
				      \]
				      (here $a=n,\ b=m$ because $\gcd(n,m)=1$; negative exponents mean division). Then
		      \[
			      \psi(\varphi(t)) = {(t^n)}^\alpha {(t^m)}^\beta = t^{\alpha n + \beta m} = t,
		      \]
			      In function fields,
			      \[
				      \psi^*(t) = x^\alpha y^\beta,
				      \qquad
				      \varphi^*(x) = t^n,\ \varphi^*(y)=t^m,
			      \]
			      so
			      \[
				      (\varphi^* \circ \psi^*)(t) = {(t^n)}^\alpha {(t^m)}^\beta = t^{\alpha n + \beta m} = t.
			      \]
			      Also
			      \[
				      (\psi^* \circ \varphi^*)(x)=x,\qquad (\psi^* \circ \varphi^*)(y)=y,
			      \]
			      so
				      \[
					      \psi^* \circ \varphi^*=\mathrm{id}_{k(V(x^b-y^a))}.
				      \]
			      Hence $\psi,\varphi$ are inverse rational maps
			      ([Definition, \S 7, p.25, Algebraic Geometry, 2026]).
	      \end{solution}

\section{}
	      \begin{enumerate}[label=(\alph*)]
			      \item Let $S=V(x^2+y^2-1)$ be the circle and $H=V(wz-1)$ be the hyperbola.
			            Show that either $S\cong H$ or $S\cong\mathbb{A}^1$.
		            \begin{solution}
			            \textbf{Case $\operatorname{char} k \neq 2$:}\quad
			            Since $k$ is algebraically closed, choose $i \in k$ with $i^2 = -1$. Define
			            \[
				            \Phi \colon S \to H, \qquad (x, y) \mapsto (x + iy,\, x - iy).
			            \]
			            On $S$: $(x + iy)(x - iy) = x^2 + y^2 = 1$, so $\Phi$ maps into $H$. The inverse is
			            \[
				            \Phi^{-1} \colon H \to S, \qquad (w, z) \mapsto \left(\frac{w + z}{2},\, \frac{w - z}{2i}\right),
			            \]
				            which is well-defined since $\operatorname{char} k \neq 2$.
				            One verifies
				            \[
					            \Phi^{-1}\circ\Phi=\mathrm{id}_S,\qquad
					            \Phi\circ\Phi^{-1}=\mathrm{id}_H.
				            \]
				            Both maps are morphisms, so $S \cong H$.

			            \textbf{Case $\operatorname{char} k = 2$:}\quad
				            In characteristic $2$,
				            \[
					            x^2+y^2-1={(x+y)}^2+1={(x+y+1)}^2,
				            \]
				            since ${(x+y+1)}^2=x^2+y^2+1$ and $-1=1$.
				            Therefore
			            \[
				            S = V({(x + y + 1)}^2) = V(x + y + 1),
			            \]
				            which is a line in $\mathbb{A}^2$.
				            The map
				            \[
					            t\mapsto (t,t+1)
				            \]
				            is an isomorphism $\mathbb{A}^1\xrightarrow{\sim}S$ with inverse $(x,y)\mapsto x$.
				            So $S\cong\mathbb{A}^1$.
			            \end{solution}

			      % \subqdash{}

			      \item Let $k$ have characteristic $p$.
			            Show that
			            \[
				            \varphi:\mathbb{A}^1\to\mathbb{A}^1,\qquad t\mapsto t^p
			            \]
			            is a bijection, and show that $\varphi$ is not a birational equivalence.
			            \begin{solution}
				            \textbf{Bijection:}

				            \emph{Surjectivity:}
				            For any $a\in k$, the polynomial $t^p-a$ has a root in $k$,
				            since $k$ is algebraically closed.

				            \emph{Injectivity:}
				            If $s^p=t^p$, then
				            \[
					            {(s-t)}^p=s^p-t^p=0,
				            \]
				            because $\binom{p}{i}=0$ in characteristic $p$ for $0<i<p$.
				            Hence $s=t$.

			      \textbf{Not a birational equivalence:}\quad
			      By [Corollary 7.3, \S 7, p.26, Algebraic Geometry, 2026],
			      irreducible affine varieties are birational iff their function fields are $k$-isomorphic.
			      Here
			      \[
				      \varphi^* \colon k(t)\to k(t),\qquad f(t)\mapsto f(t^p),
			      \]
			      and
			      \[
				      \operatorname{im}(\varphi^*)=k(t^p)\subsetneq k(t).
			      \]
			      Put $s:=t^p$. In $k(s)[X]$, consider
			      \[
				      F(X):=X^p-s.
			      \]
			      By Eisenstein in $k[s][X]$ with prime element $s$, $F$ is irreducible in $k(s)[X]$.
			      Hence the minimal polynomial of $t$ over $k(t^p)=k(s)$ has degree $p$.
			      Moreover
			      \[
				      [k(t):k(t^p)]=p>1,
			      \]
			      so $\varphi^*$ is not surjective.
			      Therefore $\varphi^*$ is not an isomorphism, hence $\varphi$ is not birational.
		            \end{solution}
	      \end{enumerate}

\section{}
Consider the cubic curve $C := V(y^2 - x^3 - x) \subseteq \mathbb{A}^2$.
	      \begin{enumerate}[label=(\alph*)]
		      \item Prove that $C$ is irreducible.
		            \begin{solution}
				            View $F=y^2-x^3-x$ as an element of $k[x][y]$.
				            If $F$ were reducible in $k[x,y]$, then since $\deg_y F=2$, it would factor as
			            \[
				            F = (y - p(x))(y - q(x))
			            \]
				            for some $p,q\in k[x]$, giving
				            \[
					            p+q=0,\qquad pq=-(x^3+x).
				            \]
				            Thus $q=-p$ and $p^2=x^3+x$.
				            But $\deg(p^2)=2\deg p$ is even, while $\deg(x^3+x)=3$ is odd.
				            Contradiction. Therefore $F$ is irreducible.
			            \end{solution}

			      % \subqdash{}

		      \item Find the domain of definition of the rational map
		            \[
			            \varphi \colon C \dashrightarrow \mathbb{A}^1,\qquad \varphi(x,y)=x/y.
		            \]
			            \begin{solution}
			            On $C$, $y^2=x^3+x=x(x^2+1)$, hence
			            \[
				            \frac{x}{y}
				            =
				            \frac{x}{y}\cdot\frac{y}{y}
				            =
				            \frac{xy}{y^2}
				            =
				            \frac{xy}{x(x^2+1)}
				            =
				            \frac{y}{x^2+1}.
			            \]
			            So $\varphi$ is regular on
			            \[
				            C\setminus V(y)
				            \quad\text{and}\quad
				            C\setminus V(x^2+1),
			            \]
			            hence on
			            \[
				            C\setminus V(y,x^2+1).
			            \]

			            Let $P\in C$ with
			            \[
				            y{(P)}=0,\qquad {x(P)}^2+1=0.
			            \]
			            Let $\mathcal O_{C,P}$ be the local ring.
			            Since ${x(P)}^2=-1$, we have $x(P)\neq 0$, so $x$ is a unit in $\mathcal O_{C,P}$.\footnote{In a local ring, an element is a unit iff its value at $P$ is nonzero.}
			            If $\varphi$ were regular at $P$, then
			            \[
				            \frac{x}{y}\in\mathcal O_{C,P},
			            \]
			            so there exists $r\in\mathcal O_{C,P}$ with
			            \[
				            x=yr.
			            \]
			            But $x$ is a unit, hence $y$ would be a unit (product of two elements equals a unit).\footnote{If $ab$ is a unit, then both $a$ and $b$ are units.}
			            This is impossible because $y(P)=0$, i.e. $y\in \mathfrak m_P$.
			            Therefore $\varphi$ is not regular at $P$.

			            Thus
			            \[
				            \operatorname{Dom}(\varphi)=C\setminus V(y,x^2+1).
			            \]
			            Explicitly:
			            \[
				            \begin{cases}
					            (i,0),\ (-i,0), & \operatorname{char}k\neq 2,\\
					            (1,0), & \operatorname{char}k=2.
				            \end{cases}
			            \]
		            \end{solution}

			      % \subqdash{}

			      \item Now consider $C' := V(y^2 - x^3 - x^2)\subseteq\mathbb{A}^2$.
			            The same formula defines a rational map $\varphi:C'\dashrightarrow\mathbb{A}^1$.
			            Find a dominant rational map $\psi:\mathbb{A}^1\dashrightarrow C'$ with
			            \[
				            \varphi\psi=\mathrm{id}_{\mathbb{A}^1}.
			            \]
			            \begin{solution}
			            Let
			            \[
				            \psi(t)=(a(t),b(t)),
				            \qquad
				            \frac{a}{b}=t,
				            \qquad
				            a=tb.
			            \]
			            Since $C':y^2=x^3+x^2$,
			            \begin{align*}
				            b^2
				            &=a^3+a^2 \\
				            &={(tb)}^3+{(tb)}^2 \\
				            &=t^3b^3+t^2b^2.
			            \end{align*}
			            Hence
			            \[
				            b^2(1-t^2)=t^3b^3.
			            \]
			            Choose
			            \[
				            b=\frac{1-t^2}{t^3},
				            \qquad
				            a=tb=\frac{1-t^2}{t^2}.
			            \]

			            Define the rational map
			            \[
				            \psi \colon \mathbb{A}^1 \dashrightarrow C', \qquad t \mapsto \left(\frac{1 - t^2}{t^2},\, \frac{1 - t^2}{t^3}\right).
			            \]

				            \emph{Verification}: $\varphi(\psi(t)) = \dfrac{(1-t^2)/t^2}{(1-t^2)/t^3} = t$.\,\checkmark{}
				            $\psi(t)\in C'$:
			            \begin{align*}
				            y^2-x^3-x^2
				            &= \frac{{(1-t^2)}^2}{t^6}
				            - \frac{{(1-t^2)}^3}{t^6}
				            - \frac{{(1-t^2)}^2}{t^4} \\
				            &= \frac{{(1-t^2)}^2\bigl(1-(1-t^2)-t^2\bigr)}{t^6} \\
				            &= 0.
			            \end{align*}
			            \checkmark{}


			            Also, $C'$ is irreducible:
			            if $y^2-x^3-x^2$ were reducible in $k[x,y]$, then
			            \[
				            y^2-x^3-x^2=(y-p(x))(y-q(x))
			            \]
			            for some $p,q\in k[x]$, so $q=-p$ and
			            \[
				            p^2=x^3+x^2.
			            \]
			            But $\deg(p^2)$ is even, whereas $\deg(x^3+x^2)=3$ is odd, contradiction.

			            \emph{Dominance:}
			            For $y\neq 0$, set
			            \[
				            t=\frac{x}{y}.
			            \]
			            Then
			            \[
				            \psi(t)=(x,y).
			            \]
			            So
			            \[
				            C'\setminus V(y)\subseteq \operatorname{im}(\psi).
			            \]
			            Since $C'\setminus V(y)$ is dense in irreducible $C'$, $\psi$ is dominant.
		            \end{solution}
	      \end{enumerate}

\end{document}
