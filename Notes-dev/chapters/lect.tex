\documentclass[../main.tex]{subfiles}
\usepackage{csquotes}
\usepackage{fontspec}
\setmonofont{FreeMono}
\begin{document}

\section{Lecture 2}
%def Quasi-Projective Variety
A \textbf{quasi-projective variety} is a subset of projective space that


\begin{example}
$\mathbb{A}^n$ is birational to $\mathbb{P}^n$ via the map
\[(x_1, \ldots, x_n) \mapsto [1 : x_1 : \ldots : x_n]\]
with inverse
\[[x_0 : x_1 : \ldots : x_n] \mapsto \left(\frac{x_1}{x_0}, \ldots, \frac{x_n}{x_0}\right)\]
(not defined when $x_0 = 0$).
check these are rational inverses
\end{example}
\begin{lemma}
  Two irreducible quasi-projective varieties $V,W$ are birational if and only if there are open subsets $A \subset V$ and $B \subset W$ such that $A$ and $B$ are isomorphic (as quasi-projectivevarieties).
\end{lemma}
\begin{proof}
  Given $f: A \to B$ with inverse $g: B \to A$ isomorphisms, we can extend to rational maps $\varphi:V \dashrightarrow W$ and $\psi  : W \dashrightarrow V$ by defining them to be undefined outside of $A$ and $B$ respectively. These are clearly rational inverses.
  Open nonempty subsets of irreducible varieties are dense, so these are indeed birational maps.
  \[
  A_1 := dom(\varphi)\subset V
  \]
  \[B_1 := dom(\psi) \subset W
  \]
  % --- converse ---
  \[
  A= \varphi^{-1}|_{A_1}(B_1) \subset A_1
  \]
  \[
  B = \psi^{-1}|_{B_1}(A_1) \subset B_1
  \]
\end{proof}
If $V$ be a quasi-projective variety, a \textbf{rational function} on $V$ is a rational map from $V$ to $\mathbb{A}^1$. The set of all rational functions on $V$ is denoted $K(V)$, called the \textbf{function field} of $V$.\\
There is the same as a rational map $\phi: V \dashrightarrow \mathbb{P}^1$ 
\end{document}
